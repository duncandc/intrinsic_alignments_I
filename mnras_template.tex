%%%%%%%%%%%%%%%%%%%%%%%%%%%%%%%%%%%%%%%%%%%%%%%%%%
% Basic setup. Most papers should leave these options alone.
\documentclass[fleqn,usenatbib]{mnras}

% MNRAS is set in Times font. If you don't have this installed (most LaTeX
% installations will be fine) or prefer the old Computer Modern fonts, comment
% out the following line
\usepackage{newtxtext,newtxmath}
% Depending on your LaTeX fonts installation, you might get better results with one of these:
%\usepackage{mathptmx}
%\usepackage{txfonts}

% Use vector fonts, so it zooms properly in on-screen viewing software
% Don't change these lines unless you know what you are doing
\usepackage[T1]{fontenc}
\usepackage{ae,aecompl}


%%%%% AUTHORS - PLACE YOUR OWN PACKAGES HERE %%%%%

% Only include extra packages if you really need them. Common packages are:
\usepackage{graphicx}	% Including figure files
\usepackage{amsmath}	% Advanced maths commands
\usepackage{amssymb}	% Extra maths symbols

%%%%%%%%%%%%%%%%%%%%%%%%%%%%%%%%%%%%%%%%%%%%%%%%%%

%%%%% AUTHORS - PLACE YOUR OWN COMMANDS HERE %%%%%

% Please keep new commands to a minimum, and use \newcommand not \def to avoid
% overwriting existing commands. Example:
%\newcommand{\pcm}{\,cm$^{-2}$}	% per cm-squared

%%% formatting
\newcommand{\beq}{\begin{eqnarray}}
\newcommand{\eeq}{\end{eqnarray}}

%%% halo model
\newcommand{\tpcf}[1]{\xi_{\rm #1}}
\newcommand{\tpcftwo}[2]{\xi_{\rm #1}^{\rm #2}}


%%% math
\newcommand{\dd}{\rm d}
\newcommand{\mean}[1]{\langle #1 \rangle}
\newcommand{\meantwo}[2]{\langle #1 \vert #2 \rangle}

%%%%%%%%%%%%%%%%%%%%%%%%%%%%%%%%%%%%%%%%%%%%%%%%%%

%%%%%%%%%%%%%%%%%%% TITLE PAGE %%%%%%%%%%%%%%%%%%%

% Title of the paper, and the short title which is used in the headers.
% Keep the title short and informative.
\title[An Empirical Model for Intrinsic Alignments I]{An Empirical Model for Intrinsic Alignments I: Insights from Dark Matter Only Simulations}

% The list of authors, and the short list which is used in the headers.
% If you need two or more lines of authors, add an extra line using \newauthor
\author[D. Campbell]{
Duncan Campbell,$^{1}$\thanks{E-mail: duncanc@andrew.cmu.edu}
Andrew Hearin $^{2}$
\\
% List of institutions
$^{1}$McWilliams Center for Cosmology, Department of Physics, Carnegie Mellon University, Pittsburgh, PA 15213, USA \\
$^{2}$Argonne National Laboratory, Lemont, IL 60439, USA
\\
}

% These dates will be filled out by the publisher
\date{Accepted XXX. Received YYY; in original form ZZZ}

% Enter the current year, for the copyright statements etc.
\pubyear{2018}

% Don't change these lines
\begin{document}
\label{firstpage}
\pagerange{\pageref{firstpage}--\pageref{lastpage}}
\maketitle

% Abstract of the paper
\begin{abstract}
Galaxies exhibit alignments with both close neighbours and the large scale structure.  The nature of intrinsic alignments is interesting for both testing galaxy formation models and in cosmological probes of dark energy.  Any intrinsic alignment between galaxies may be a significant source of systematic error in cosmological probes of dark energy which rely on weak lensing.  Because upcoming surveys rely on photometric redshifts, some lens and source galaxies may be physically close, and any intrinsic alignment between galaxies will masquerade as a lensing signal.  In this series of papers, we present a flexible, empirical, halo based model of intrinsic alignments of galaxies.  In this first paper, we examine the necessary components and phenomenology of such a model by examining the alignments between (sub-)haloes in a cosmological dark matter only simulation.  The primary assumption of our model is that galaxies exhibit an alignment with their host dark matter (sub-)halo.  By modelling the misalignment between galaxies orientations and their host halo, we show that (sub-)halo two-point position and shape correlation functions can be accurately reproduced down to $0.1~h^{-1}{\rm Mpc}$.  To model the small scale alignments of dark matter sub-haloes, a radial alignment model is sufficient when combined with a model for satellite anisotropy.   \end{abstract}

% Select between one and six entries from the list of approved keywords.
% Don't make up new ones.
\begin{keywords}
keyword1 -- keyword2 -- keyword3
\end{keywords}

%%%%%%%%%%%%%%%%%%%%%%%%%%%%%%%%%%%%%%%%%%%%%%%%%%

%%%%%%%%%%%%%%%%% BODY OF PAPER %%%%%%%%%%%%%%%%%%

\section{Introduction}

We extend the halo based model for alignments presented in \citet{Schneider:2010cg}.

\section{Methods}

\subsection{Observables}

\subsubsection{(Sub-)halo Shapes}
Give a description of the observables and relevant quantities and each is calculated in Halotools.

We assume that (sub-)haloes and galaxies can be modelled as 3D ellipsoidals specified by the eigenvectors, $\hat{e}_a$, $\hat{e}_b$, and $\hat{e}_c$ and eigenvalues, $\lambda_a$, $\lambda_b$, and $\lambda_c$, of the inertia tensor of the mass distribution. 

We use the reduced inertia tensor defined as:
%
\begin{equation}
\tilde{\bf I}_{ij} = \frac{\sum m_n \frac{x_{ni} x_{nj}}{r_{n}^2}}{\sum m_n}
\end{equation}
%
where
%
\begin{equation}
r_{n}^2 = \sum x_{ni}^2
\end{equation}
%
is the distance between the centre of mass and the $n^{\rm th}$ particle in the system.

$\hat{e}_{\lambda}({\bf x})$ is the major, intermediate, or minor axis ($\lambda=A,B,C$) of the galaxy or halo.  

The ellipticity-direction (ED) correlation function is defined as: 
\begin{equation}
\omega(r) = \langle |\hat{e}({\bf x}) \cdot \hat{r}({\bf x})|^2 \rangle -\frac{1}{3}
\end{equation}

The ellipticity-ellipticity (EE) correlation function is defined as: 
\begin{equation}
\eta(r) = \langle |\hat{e}({\bf x}) \cdot \hat{e}({\bf x}+{\bf r})|^2 \rangle -\frac{1}{3}
\end{equation}


\subsection{Halo Model}
Presented halo model equations needed to model the EE and ED correlation functions.

\beq
\tpcf{gg}(r) = \tpcftwo{gg}{1h}(r) + \tpcftwo{gg}{2h}(r)
\eeq

\beq
\tpcftwo{gg}{1h}(r) = \tpcftwo{cs}{1h}(r) + \tpcftwo{ss}{1h}(r) 
\eeq

\beq
\tpcftwo{gg}{2h}(r) =  \tpcftwo{cc}{1h}(r) + \tpcftwo{cs}{1h}(r) + \tpcftwo{ss}{1h}(r) 
\eeq

\begin{align}
\tpcftwo{gg}{2h}(r) =   \frac{1}{\bar{n}_{\rm g}^{2}}\int\dd M_1 \int\dd M_2 & \frac{\rm dn}{\dd M_1}\frac{\rm dn}{\dd M_2} \tpcf{hh}(r\vert M_1, M_2) \\& \times  \meantwo{N_{\rm g}}{M_1} \meantwo{N_{\rm g}}{M_2}. 
\end{align}

\begin{align}
\tpcftwo{gg}{1h}(r) =  \frac{1}{\bar{n}_{\rm g}^{2}} \int\dd M \frac{\rm dn}{\dd M} \int\dd^{3}& x\lambda(\vec{x}\vert M)\lambda(\vec{x}+\vec{r}\vert M)\\ & \times  \meantwo{N_{\rm g}(N_{\rm g}-1)}{M}
\end{align}

%\int\dd M_1 \int\dd M_2 & \frac{\rm dn}{\dd M_2} \tpcf{hh}(r\vert M_1, M_2) \\& \times \mean{N_{\rm g}(M_1)N_{\rm g}(M_2)}. 

\section{Results}
Present the alignment models and the affect of varying the model on the EE and ED correlation functions

\subsection{Modelling Misalignments}
We use a Dimroth-Watson distribution to model the misalignment between subhalo orientation and the radial vector between the host-halo centre and the subhalo.   
%
\begin{equation}
P(\theta,\phi) = \frac{B(\kappa)}{2\pi}e^{-\kappa\cos^2(\theta)}\sin(\theta)\mathrm{d}\theta\mathrm{d}\phi
\label{eq:watson}
\end{equation}
%
where the normalization factor is given by:
%
\begin{equation}
B(\kappa) = \frac{1}{2}\int_0^1 e^{-\kappa t^2}\mathrm{d}t
\end{equation}
%

\subsection{1-Halo Intrinsic Alignments}

We model the number density profile of satellites in haloes as a triaxial NFW profile \citep{Jing:2002bs}.
%
\begin{equation}
\rho(R) = \frac{\rho_c}{\frac{R}{R_s}\left( 1+\frac{R}{R_s} \right)^2}
\end{equation}
%
where the relation between the Cartesian and elliptical coordinates are given by:
%
\begin{align}
x &= r\sin(\theta)\cos(\phi) = R\frac{a}{c}\sin(\Theta)\cos(\Phi) \nonumber \\
y &= r\sin(\theta)\sin(\phi) = R\frac{b}{c}\sin(\Theta)\sin(\Phi) \nonumber \\
z &= r\cos(\theta) = R \cos(\Theta)
\end{align}
%
where $a,b$, and $c$ are the length of the axes of the ellipsoid.

\subsubsection{Alignment Vector Model}

\subsubsection{Satellite Anisotropy}

\subsection{2-Halo Intrinsic Alignments}

\subsubsection{Alignment Vector Model}

\subsubsection{Satellite Anisotropy}

\subsection{Connecting to the Linear Alignment Model}

\section{Conclusions}

\section*{Acknowledgements}

%%%%%%%%%%%%%%%%%%%%%%%%%%%%%%%%%%%%%%%%%%%%%%%%%%

%%%%%%%%%%%%%%%%%%%% REFERENCES %%%%%%%%%%%%%%%%%%

\bibliographystyle{mnras}
\bibliography{bib}

%%%%%%%%%%%%%%%%%%%%%%%%%%%%%%%%%%%%%%%%%%%%%%%%%%

%%%%%%%%%%%%%%%%% APPENDICES %%%%%%%%%%%%%%%%%%%%%

%%%%%%%%%%%%%%%%%%%%%%%%%%%%%%%%%%%%%%%%%%%%%%%%%%


% Don't change these lines
\bsp	% typesetting comment
\label{lastpage}
\end{document}